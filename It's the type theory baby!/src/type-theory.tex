\documentclass{beamer}

\mode<presentation>
{
  \usetheme{Montpellier}
  \usecolortheme[light]{solarized}

  % or ...

  \setbeamercovered{transparent}
  % or whatever (possibly just delete it)
}

\usepackage[T1]{fontenc}
\usepackage{url}
\usepackage{ucs}
\usepackage[utf8x]{inputenc}
\usepackage[english, italian]{babel}
\setcounter{secnumdepth}{2}
\usepackage{eurosym}
\usepackage{graphics}
\usepackage{fancyhdr}
\usepackage{setspace}
\usepackage{graphicx}
\usepackage{fancybox}
\usepackage{marginnote}
\usepackage{tabularx}
\usepackage{multirow}
\usepackage{colortbl}
\usepackage{booktabs}
\usepackage{siunitx}
\sisetup{group-separator = {.}}
\sisetup{group-minimum-digits = 4}
\sisetup{output-decimal-marker={,}}
\usepackage{lettrine}
\usepackage{ulem}
\usepackage{microtype}
\usepackage{alltt}
\usepackage{wrapfig}
\usepackage{listings}
\lstset{% general command to set parameter(s)
    basicstyle=\tiny, % print whole listing small
    keywordstyle=\color{black}\bfseries\underbar, % underlined bold black keywords
    identifierstyle=,% nothing happens
    commentstyle=\color{white}, % white comments
    stringstyle=\ttfamily,% typewriter type for strings
    showstringspaces=false}% no special string spaces

\usepackage{mathpazo}
%\usepackage[scaled=.95]{helvet}
\usepackage{palatino}
\usepackage{courier}

\usepackage{soul}
\usepackage[italian]{varioref}
\usepackage{hyperref}
\hypersetup{%
    colorlinks  = true,
    linktoc     = all,
    unicode     = true,
    citecolor   = royalpurple,
    linkcolor   = blue,
    urlcolor    = red,
}

\title[Type theory] % (optional, use only with long paper titles)
{ It's the type theory baby! }

\subtitle{A gentle introduction to type theory (well, not so gentle
\dots)}

\author[Michele Finelli] % (optional, use only with lots of authors)
{Michele Finelli\\
m@biodec.com\\
BioDec}
% - Use the \inst command only if there are several affiliations.
% - Keep it simple, no one is interested in your street address.

\date{}

\subject{Talks}
% This is only inserted into the PDF information catalog. Can be left
% out. 

% If you have a file called "university-logo-filename.xxx", where xxx
% is a graphic format that can be processed by latex or pdflatex,
% resp., then you can add a logo as follows:

\pgfdeclareimage[height=0.5cm]{university-logo}{biodec_rosso}
\logo{\pgfuseimage{university-logo}}

% Delete this, if you do not want the table of contents to pop up at
% the beginning of each subsection:
\AtBeginSection[]
{
  \begin{frame}<beamer>
    \frametitle{Index}
    \tableofcontents[currentsection,currentsubsection]
  \end{frame}
}

\AtBeginSubsection[]
{
  \begin{frame}<beamer>
    \frametitle{Index}
    \tableofcontents[currentsection,currentsubsection]
  \end{frame}
}

% If you wish to uncover everything in a step-wise fashion, uncomment
% the following command: 

%\beamerdefaultoverlayspecification{<+->}

\begin{document}

\begin{frame}
  \titlepage
\end{frame}

\begin{frame}
  \frametitle{Index}
  \tableofcontents
  % You might wish to add the option [pausesections]
\end{frame}

\section{What is a type ?} % (fold)
\label{sec:What is a type ?}

\begin{frame}

    \frametitle{This is a talk about types}

    So \dots what is a type in programming languages ?

    \begin{enumerate}
    
        \uncover <2->{\item A type is a \textbf{property}.}
        
        \uncover <3->{\item A type is \textbf{assigned to a term}.}
        
        \uncover <4->{\item So a type determines that a term
            (\textit{i.e.} a ``program fragment'') has a well defined
        \textbf{meaning}.}
    
    \end{enumerate}

\end{frame}

\begin{frame}

    \frametitle{What do we mean by ``meaning'' ?}

    So \dots which are the \textbf{acceptable meanings} of a program
    fragment ? Which of the following is an \textbf{acceptable property} ?

    \begin{enumerate}
    
        \uncover <2->{\item Is it being an IEEE 754-2008 float ?}
        
        \uncover <3->{\item Is the assurance that a given memory
            location will not be changed ?}
        
        \uncover <4->{\item Is the statement that a given programming
            structure has a certain size ?}
        
        \uncover <5->{\item What about this ? --- N are the natural
                numbers, Perm is a permutation.
            \begin{align*}
                \forall n, \forall f: N \rightarrow N, \exists \pi \in \mbox{Perm}, \forall 1 \leq i,j \leq n & \\
                i \leq j \implies f(\pi(i)) \leq f(\pi(j)) & 
            \end{align*}
            }
    
    \end{enumerate}

\end{frame}

\begin{frame}

    \frametitle{Meaningful answers}

    \begin{enumerate}
    
        \uncover <2->{\item Is being an IEEE 754-2008 float a
            \textbf{property} ?} 
            \uncover <3->{\textbf{Yes}. It is a typical variable typing
            judgement.}
        
        \uncover <4->{\item Is the assurance that a given memory
            location will not be changed, a \textbf{property} ?}
            \uncover <4->{\textbf{Yes}. It is a safety property (not so easy to
            express).}
        
    \end{enumerate}

\end{frame}

\begin{frame}

    \frametitle{Meaningful answers}

    \begin{enumerate} \setcounter{enumi}{2}
        
        \uncover <2->{\item Is the statement that a given programming
            structure has a certain size, a \textbf{property} ?} 
            \uncover <3->{\textbf{Yes}. It could be a dependent type judgement,
            for example.}
        
        \uncover <4->{\item What about this ? --- N are the natural
                numbers, Perm is a permutation.
            \begin{align*}
                \forall n, \forall f: N \rightarrow N, \exists \pi \in \mbox{Perm}, \forall 1 \leq i,j \leq n & \\
                i \leq j \implies f(\pi(i)) \leq f(\pi(j)) & 
            \end{align*}
            }

            \uncover <5->{\textbf{Yes again}. It states that there is an
            algorithm that sorts the natural numbers.}
        
    \end{enumerate}

\end{frame}

\begin{frame}

    \frametitle{A property is a logical formula}

    \begin{itemize}
    
        \uncover <2->{\item A type judgement about a program fragment is
            \textit{simply} a \textbf{logical formula} that is satisfied
            by that program.}
    
        \uncover <3->{\item But a program does not ``stand still'' \dots
            it \textit{runs}, it \textit{gets executed}: a formula
            \textit{au contraire} is \textbf{always} true or false, how do
            we reconcile this \textbf{static versus dynamic} conundrum ?}
    
    \end{itemize}

\end{frame}

\begin{frame}

    \frametitle{Formulas are eternal !}

    \begin{itemize}
 
        \uncover <2->{\item Since the \textbf{provability} of a logical
                formula does not change with time, we must assume that
                also the meaning of the fragment of code is
                \textbf{absolutely determined} and that it does not
            depend on being executed.} 
            
        \uncover <3->{\item \textbf{Execution} is just the reduction of
            the code to something simpler but equivalent: the
            \textbf{result of the computation}.}

    \end{itemize}

\end{frame}

\begin{frame}

    \frametitle{Propositions as types is a powerful notion}

    Providing a type to a term subsumes some properties of the code
    fragment to which the type is attached:

    \begin{itemize}
    
        \uncover <2->{\item the code must be \textit{pure},
            \textit{i.e.} without \textbf{side effects} since the
        meaning of the code can not depend on something external to the
        program;}
    
        \uncover <3->{\item since the formula is completely determined
                by its variables, we have a notion of
            \textbf{functionality} of the related code;}
    
        \uncover <4->{\item we have a notion of \textbf{immutability}.}
    
    \end{itemize}

\end{frame}

\begin{frame}

    \frametitle{Sounds familiar ?}

    \begin{itemize}
    
        \uncover <2->{\item Purity, or ``no side effects'' ?} 
    
        \uncover <3->{\item Functions as the building blocks.}
    
        \uncover <4->{\item Immutable values.}
    
    \end{itemize}

    \uncover <5->{\Large Yes ! I am talking about \textbf{functional
    programming} !}

\end{frame}

% section What is a type ? (end)

\section{Short history of type theory} % (fold)
\label{sec:Short history of type theory}

\begin{frame}

    \frametitle{Or, how we got the \textit{propositions as types}
    interpretation}

    This section is a brief outline of type theory, mainly focused on
    the the branch that studied the interpretation of:

    \begin{itemize}
    
        \uncover <2->{\item propositions as types,}
    
        \uncover <3->{\item proofs as programs,}
    
        \uncover <4->{

            \begin{center}
        
                Program : Type $\Leftrightarrow$ Proof : Proposition
                
            \end{center}
        
        }

        \uncover <5->{\item execution as normalization.}

    \end{itemize}

\end{frame}

\begin{frame}

    \frametitle{Propositions as types}

    \dots or ``Formulae as Types'', Curry-Howard-de Bruijn
    Correspondence, Brouwer’s Dictum, and others.

    \begin{itemize}
    
        \uncover <2->{\item The analogy between a certain kind of
            logical formulas, namely the \textbf{propositions}, and the
        meaning given to \textbf{types} is indeed a \textbf{mathematical
        theorem}.}
    
        \uncover <3->{\item It is usually referred as the \textbf{Curry
            - Howard Isomorphism}.}
    
    \end{itemize}

\end{frame}

\begin{frame}

    \frametitle{Propositions as types}

    \begin{quotation}
    
        Propositions as Types is a notion with breadth. It applies to a
        range of logics including propositional, predicate,
        second-order, intuitionistic, classical, modal, and linear. It
        underpins the foundations of functional programming, explaining
        features including functions, records, variants, parametric
        polymorphism, data abstraction, continuations, linear types, and
        session types. (Wadler, Propositions as types)
    
    \end{quotation}

\end{frame}

\begin{frame}

    \frametitle{Propositions as types}

    (Wadler, continued)

    \begin{itemize}
    
        \uncover <2->{\item \begin{quotation}
        
            Why should it be the case that intuitionistic natural
            deduction, as developed by Gentzen in the 1930s, and
            simply-typed $\lambda$-calculus, as developed by Church
            around the same time for an unrelated purpose, should be
            discovered thirty years later to be essentially identical?
        
        \end{quotation}}
    
        \uncover <3->{\item \begin{quotation}
        
            The logician Hindley and the computer scientist Milner
            independently developed the same type system, now dubbed
            Hindley-Milner.
       
        \end{quotation}}
    
        \uncover <4->{\item \begin{quotation}
        
            The logician Girard and the computer scientist Reynolds
            independently developed the same calculus, now dubbed
            Girard-Reynolds.
        
        \end{quotation}}

    \end{itemize}

\end{frame}

\begin{frame}

    \frametitle{Propositions as types}

    (Wadler, continued)

    \begin{itemize}
    
        \uncover <2->{\item \begin{quotation}
        
            In 1934, Curry observed a curious fact, relating a theory of
            functions to a theory of implication. Every type of a
            function ($A \rightarrow B$) could be read as a proposition
            ($A \supset B$), and under this reading the type of any
            given function would always correspond to a provable
            proposition. 
        
        \end{quotation}}
    
        \uncover <3->{\item \begin{quotation}
        
            Conversely, for every provable proposition there was a
            function with the corresponding type.

        \end{quotation}}
    
    \end{itemize}

\end{frame}

\begin{frame}

    \frametitle{Propositions as types}

    (Wadler, continued)

    \begin{itemize}
    
        \uncover <2->{\item \begin{quotation}
     
            In 1969, Howard circulated a xeroxed manuscript. It was not
            published until 1980, where it appeared in a Festschrift
            dedicated to Curry.  Motivated by Curry’s observation,
            Howard pointed out that there is a similar correspondence
            between natural deduction, on the one hand, and simply-typed
            $\lambda$-calculus, on the other, and he made explicit the
            third and deepest level of the correspondence (\dots), that
            simplification of proofs corresponds to evaluation of
            programs.
        
        \end{quotation}}
    
    \end{itemize}

\end{frame}

\begin{frame}

    \frametitle{Propositions as types}

    \begin{center}
        
        {\Huge Isomorphism}

    \end{center}

\end{frame}


% section Short history of type theory (end)

\section{Why types matters} % (fold)
\label{sec:Why types matters}

\begin{frame}

    \frametitle{Type systems helps}

    I will give a few examples taken from a recent research that shows
    the relevance of static typing (as opposed to dynamic typing) to
    build correct code in less time (ICSE 2014, OOPSLA 2012).

\end{frame}

\begin{frame}

    \frametitle{ICSE 2014 How Do API Documentation \dots}

    \begin{quotation}
    
        The presence of a static type system had a significant positive
        effect on development time: Subjects using the statically typed
        language required between 15 and 89 minutes less time for
        solving the task.
    
    \end{quotation}

\end{frame}

\begin{frame}

    \frametitle{OOPSLA 2012 Static Type Systems (Sometimes) \dots}

    \begin{quotation}
    
        We gave 27 subjects five programming tasks and found that the
        type systems had a significant impact on the development time:
        for three of five tasks me measured apositive impact of the
        static type system, for two tasks we measured a positive impact
        of the dynamic type system.
    
    \end{quotation}

\end{frame}

\begin{frame}

\frametitle{ICFP 2010 Experience Report: Haskell as a Reagent}

    \begin{quotation}
    
        At the end, the question was: is the extra effort needed for
        maintaining code written in two languages justified? Do we get
        any advantage out of combining two high-level, but quite
        different, languages?  As we try to show in this paper, in our
        experience the answer is affirmative, sometimes in non obvious
        ways.
    
    \end{quotation}

\end{frame}

% section Why types matters (end)

\section{Types in programming languages} % (fold)
\label{sec:Types in programming languages}

\begin{frame}

    \frametitle{Why do not we program in the only one (\textit{true}) typed language ?}

    \begin{itemize}
    
        \uncover <2->{\item Because it does not exist !}
    
        \uncover <3->{\item Historically, languages have been built
            before the theory formalized algorithms and techniques ---
            the concept of monad as a way of encapsulating
            \textit{state} in functional languages is in (Moggi,
            Information and Computation 93, 1991).} 
    
        \uncover <4->{\item If types are logical formulas we still have
            the problem of choosing the right \textbf{logical theory}.}

        \uncover <5->{\item Moreover there is not an agreement on what
            is the right amount of typing: we are full of endless
            \textit{this feature versus that feature} like \dots}
    
    \end{itemize}

\end{frame}

\begin{frame}

    \frametitle{Typed vs untyped}

    \begin{quotation}
    
        Languages that do not restrict the range of variables are called
        \textbf{untyped languages}: they do not have types or,
        equivalently, have a single universal type that contains all
        values. In these languages, operations may be applied to
        inappropriate arguments: the result may be a fixed arbitrary
        value, a fault, an exception, or an unspecified effect.
        (Cardelli, Type Systems)
    
    \end{quotation}

\end{frame}

\begin{frame}

    \frametitle{Untyped or \textit{uni}typed languages ?}

    \begin{itemize}
    
        \uncover <2->{\item Lisp is the typical untyped language \dots
            is Ruby a typed language ?}
    
        \uncover <3->{\item Or Javascript (Node) ?}
    
        \uncover <4->{\item People deal with  the issue of having a type
            system \textbf{which allows bad things to happen} saying
            that \dots}
    
    \end{itemize}

\end{frame}

\begin{frame}

    \frametitle{Static \textit{vs} dynamic}

    \dots the type system is \textit{dynamic} or that their terms are
    \textit{duck typed}.

    \begin{itemize}
    
        \uncover <2->{\item Come on ! It does not make any sense: either
            a language is typed, or it is not.}
    
        \uncover <3->{\item A language can be very useful even if it is
            not typed: this is not the point.}
    
        \uncover <4->{\item Because the real issue is that even
            (\textit{statically}) typed languages \textbf{can fail},
            because of \dots }
    
    \end{itemize}

\end{frame}

\begin{frame}

    \frametitle{Safe \textit{vs} unsafe}

    \dots \textbf{safety}. Quoting again Cardelli:

    \begin{quotation}
    
        In reality, certain statically checked languages do not ensure
        \textbf{safety}. That is, their set of forbidden errors does not
        include all untrapped errors. (\dots) For example (\dots) C has
        many unsafe and widely used features, such as pointer arithmetic
        and casting. It is interesting to notice that \textbf{the first
        five of the ten commandments for C programmers are directed at
        compensating for the weak-checking aspects of C}.  Some of the
        problems caused by weak checking in C have been alleviated in C++,
        and even more have been addressed in Java, confirming a trend away
        from weak checking.
    
    \end{quotation}

\end{frame}

\begin{frame}

    \frametitle{Explicit \textit{vs} implicit}

    \begin{itemize}
    
        \uncover <2->{\item The real burden in using types is the
            \textbf{extra work needed to specify types}: see what
            happens in C++ (even with \textit{auto}) or Java.}
    
        \uncover <3->{\item \textbf{The real advantage} is in using
            languages where types can be \textit{inferred} by the
            compiler, with none to little help from the programmer.}
    
        \uncover <4->{\item Even better if the \textit{typing relation}
            is decidable (Scala type system, for example, is known to be
            Turing complete).}
    
    \end{itemize}

\end{frame}

% section Types in programming languages (end)

\section{Conclusions} % (fold)
\label{sec:Conclusions}

\begin{frame}

    \frametitle{Do not miss an opportunity}

    \begin{itemize}
    
        \uncover <2->{\item A functional programming language without a
            type system \textbf{misses an opportunity}.}
    
        \uncover <3->{\item A type system whose term language is not
                isomorphic to the programming langage \textbf{misses an
            opportunity}.}
    
        \uncover <4->{\item Functional programming and type systems are
            two faces of the same thing: \textbf{do not use one without
        the other}.}
    
    \end{itemize}

\end{frame}

\begin{frame}

    \frametitle{Questions}

    \begin{center}
        
        {\Huge ?}

    \end{center}

\end{frame}

\begin{frame}

    \frametitle{Thanks \& see you soon \dots}

    \begin{description}
    
        \uncover <2->{\item[Thanks] for paying attention.}
    
        \uncover <3->{\item[The biggest Italian DevOps meeting] is in Bologna,
            April $10^{\mbox{th}}$, 2015.}
        
        \uncover <4->{\item[IDI2015] Incontro DevOps Italia 2015:
            \url{http://incontrodevops.it/idi2015/} }
    
    
    \end{description}

\end{frame}

% section Conclusions (end)

\end{document}
